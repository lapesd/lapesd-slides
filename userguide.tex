% Ao menos uma linguagem (brazil ou english) deveria sempre ser fornecida
\documentclass[brazil]{lapesd-slides}

\usepackage{pgfgantt}

%%%%%%%%%%%%%%%%%%%%%%%%%%%%
% Metadados
%%%%%%%%%%%%%%%%%%%%%%%%%%%%

\title[LAPESD beamer user guide]{Beamer template for LAPESD}
\subtitle{An User Guide}
\author[Dr. Trovato]{\large Ben Trovato \\{\small \texttt{ben.trovato@posgrad.ufsc.br}}}

\institute{
  \fontsize{10.5}{12.6}\selectfont 
  Programa de Pós-Graduação em Ciência da Computação\\ 
  Depto. de Informática e Estatísitca\\
  Universidade Federal de Santa Catarina - Florianópolis\\
  \vspace{1em}
  \large Orientador: Prof. Lars Thørväld
}

\date{\today}

%%%%%%%%%%%%%%%%%%%%%%%%%%%%
% Slides
%%%%%%%%%%%%%%%%%%%%%%%%%%%%

\begin{document}

\titleframe

% Você não é obrigado a colocar um sumário!
\begin{frame}{Sumário}
  \tableofcontents
\end{frame}

% Desse ponto em diante serão inseridos slides de pausa a cada \section
\showsections
\section{Introdução}

\begin{frame}[fragile]{Origem e Uso}
  \begin{itemize}
  \item Esse template funciona como uma classe \LaTeX
    \begin{itemize}
    \item O .cls é \emph{standalone}: \textbf{não depende} de arquivos externos
    \item Obtenha atualizações no \href{https://github.com/lapesd/lapesd-slides/}{GitHub}
    \item \href{https://www.overleaf.com/read/zbmmnfkmhdwz}{Template no Overleaf} 
      \begin{itemize}
      \item Cuidado: pode não estar com o último .cls
      \end{itemize}
    \end{itemize}
  \item Atualização do feito pelo Eduardo Camilo Inácio
    \begin{itemize}
    \item Novidades: Re-empacotamento, bugfixes e comandos novos
    \end{itemize}
  \end{itemize}
\end{frame}

% +========================================================================+
% |O =singlieslide só é necessário pois há um \end{frame} dentro do minted.|
% |Normalmente, um fragile simples é suficiente (e mais recomendável)      |
% +========================================================================+
%                    |
%                    v
\begin{frame}[fragile=singleslide]{Esqueleto}
\begin{minted}{latex}
\documentclass[brazil]{lapesd-slides}
\title{Beamer template for LAPESD}
\subtitle{An user guide}
\author{\large Ben Trovato}
\institute{}
\date{\today}

\begin{document}
  % .... slides ....
  
  \thanksframe
  \referencesframe{main}
  
  \begin{backup}
    % slides ...
  \end{backup}
\end{document}
\end{minted}  
\end{frame}

\begin{frame}[fragile]{Pacotes incluídos}
  \tiny
  \begin{itemize}
  \item \mla[fontsize=\tiny]|\usepackage[utf8]{inputenc}|
  \item \mla[fontsize=\tiny]|\usepackage[T1]{fontenc}|
  \item \mla[fontsize=\tiny]|\usepackage{lipsum}|
  \item \mla[fontsize=\tiny]|\usepackage{xcolor}|
  \item \mla[fontsize=\tiny]|\usepackage{booktabs}|
  \item \mla[fontsize=\tiny]|\usepackage{array}|
  \item \mla[fontsize=\tiny]|\usepackage{multirow}|
  \item \mla[fontsize=\tiny]|\usepackage{adjustbox}|
  \item \mla[fontsize=\tiny]|\usepackage[absolute,overlay]{textpos}|
  \item \mla[fontsize=\tiny]|\usepackage{hyperref}|
  \item \mla[fontsize=\tiny]|\usepackage{algorithmicx}|
  \item \mla[fontsize=\tiny]|\usepackage[noend]{algpseudocode}|
  \item \mla[fontsize=\tiny]|\usepackage{todonotes}|
  \item \mla[fontsize=\tiny]|\usepackage{soulutf8}|
  \item \mla[fontsize=\tiny]|\usepackage{babel}|
  \item \mla[fontsize=\tiny]|\usepackage{abntex2cite}| ou \mla[fontsize=\tiny]|\usepackage{natbib}| ou nenhum
  \item \mla[fontsize=\tiny]|\usepackage{pifont}|
  \item \mla[fontsize=\tiny]|\usepackage{minted}|
  \end{itemize}
\end{frame}

\begin{frame}{Makefile}
  \begin{itemize}
  \item Esse repositório inclui um Makefile
    \begin{itemize}
    \item O uso não é obrigatório
    \item Você pode compilar os slides normalmente com \texttt{pdflatex -shell-escape}
    \end{itemize}
  \item O Makefile converte SVGs dentro de \texttt{imgs/} em PDFs
  \item Veja o slide de exemplo intitulado animação
    \begin{itemize}
    \item Nesse caso o Makefile possui uma lista de frames a serem extraídos como PDFS de um SVG.
    \end{itemize}
  \end{itemize}
  
  
\end{frame}

\section{Opções da classe}

\begin{frame}{Linguagens}
  \begin{itemize}
  \item Suporte a idiomas via \texttt{babel}
  \item Passe o idioma (\texttt{brazil} ou \texttt{english}) como opção da classe \texttt{lapesd-slides}
  \end{itemize}
\end{frame}

\begin{frame}{\texttt{disabletodonotes}}
  \begin{itemize}
  \item O pacote \texttt{todonotes} fornece balõezinhos  de TODO's
  \item A opção \texttt{disabletodonotes} faz com que eles não sejam renderizados
  \end{itemize}
\end{frame}

\begin{frame}[fragile]{Slides de backup}
\begin{minted}{latex}
\begin{backup}
  \pholder{Gráfico que não coube no tempo}
  \pholder{Lamentações sobre um detalhe técnico qualquer}
\end{backup}
\end{minted}
  \begin{itemize}
  \item Opção \texttt{showbackup} faz com que o conteúdo do ambiente seja renderizado
  \item Opção \texttt{hidebackup} impede renderização do conteúdo do ambiente
  \end{itemize}
\end{frame}

\begin{frame}[fragile]{Citações}
  \begin{itemize}
  \item Três opções:
    \begin{itemize}
    \item \texttt{abntexcite}: Carrega e configura o \texttt{abntex2cite}
    \item \texttt{natbib}: Carrega e configura o \texttt{natbib}
    \item \texttt{nobib}: Não carrega nada
    \end{itemize}
  \item Macros \mla|\cite|, \mla|\citet|, \mla|\citep| e \mla|\citeonline| sempre estarão disponíveis, não importa se for escolhido \texttt{abntex2cite} ou \texttt{natbib}
  \end{itemize}
\end{frame}


\section{Novos comandos e Ambientes}

\begin{frame}[fragile,fragile]{\texttt{\textbackslash{}titleframe}}
  \begin{center}
    \mla|\titleframe|
  \end{center}
  \begin{itemize}
  \item Gera o slide de título
  \item Não use dentro de um ambiente \texttt{frame}
  \end{itemize}
\end{frame}

\begin{frame}[fragile]{\texttt{\textbackslash{}showsections}}
  \begin{center}
    \mla|\showsections|
  \end{center}
  \begin{itemize}
  \item Passa a gerar slides de ``pausa'' a cada \mla|\section|
  \item Os slides contem apenas o título da seção, sem sidebar
  \end{itemize}
\end{frame}

\begin{frame}[fragile]{\texttt{\textbackslash{}stopcountingframes}}
  \begin{center}
    \mla|stopcountingframes|
  \end{center}
  \begin{itemize}
  \item A partir desse ponto slides deixam de ser contabilizados
  \item O último slide será o slide n/n
  \item O próximo slide será o slide $n+1$/n
  \end{itemize}
\end{frame}

\begin{frame}[fragile]{\texttt{\textbackslash{}thanksframe}}
  \begin{center}
    \mla|\thanksframe[linha1][linha2]|
  \end{center}
  \begin{itemize}
  \item Gera slide de agradecimento, usando o estilo do título
    \begin{itemize}
    \item Não use dentro do ambiente \texttt{frame}
    \end{itemize}
  \item Argumentos \emph{default}:
    \begin{itemize}
    \item \texttt{linha1}: Obrigado! / Thank You!
    \item \texttt{linha2}: Perguntas? / Questions? 
    \end{itemize}
  \end{itemize}
\end{frame}

\begin{frame}[fragile]{\texttt{\textbackslash{}referencesframe}}
  \begin{center}
    \mla|\referencesframe{bibfiles}|
  \end{center}
  \begin{itemize}
  \item Gera slides de referências
    \begin{itemize}
    \item Não use dentro do ambiente \texttt{frame}
    \end{itemize}
  \item Pode gerar mais de um slide
  \item \texttt{bibfiles} é repassado para \mla|\bibliography|
  \end{itemize}
\end{frame}

\begin{frame}[fragile]{\texttt{\textbackslash{}pholder}}
  \begin{center}
    \mla|\pholder[todonote]{titulo}|
  \end{center}
  \begin{itemize}
  \item Gera um slide \emph{placeholder} com o título fornecido
    \begin{itemize}
    \item \texttt{todonote} texto do \mla|\todo[inline]{}| deixado no corpo do slide. O default é ``Fazer''
    \end{itemize}
  \item Útil para criar esqueletos de apresentação 
  \end{itemize}
\end{frame}

\begin{frame}[fragile]{\texttt{\textbackslash{}addfig}}
  \begin{center}
    \begin{minipage}{.9\linewidth}
\begin{minted}[linenos=disable]{latex}
\addfig[texto][opts_graphicx]{arquivo.ext}
\addfiglw[texto]{arquivo.ext}
\end{minted}
    \end{minipage}
  \end{center}
  \begin{itemize}
  \item Atalho para incluir figuras
    \begin{itemize}
    \item A figura ficará centralizada
    \item Haverá espaçamento de \texttt{.5em} (meia linha) antes e depois
    \item Se \texttt{texto} for dado, será renderizado abaixo da figura, centralizado
    \item \texttt{opts\_graphicx} é redirecionado para o \mt|\includegraphics|
    \end{itemize}
  \item \mt|addfiglw| tem o mesmo comportamento, mas usa \mt|width=\linewidth| como \texttt{opts\_graphicx}
  \end{itemize}
\end{frame}

\begin{frame}[fragile]{\texttt{\textbackslash{}\textbf{t}cite*}}
  \begin{center}
    \mla|\tcitet[]{}|, \mla|\tcitep[]{}|, \mla|\tcite[]{}| e \mla|\tciteonline[]{}|
  \end{center}
  \begin{itemize}
  \item Equivalentes às suas contra-partes sem o \textbf{\texttt{t}}
  \item Não são definidos se o \mla|\cite| correspondente não existir
  \item Ao usar a opção de classe abntexcite, \mla|\citep| e \mla|\citet| são definidos a partir de \mla|\cite| e \mla|\citeonline|
  \item Ao usar a opção de classe natbib, \mla|\citeonline| é definido a partir de \mla|\citet|
  \end{itemize}
\end{frame}


\section{Exemplos}

% Atalho para criar esqueletos dos slides
\pholder[Slide criado com \mt|\pholder|]{Slide placeholder}

\begin{frame}{Citations}
  \begin{itemize}
  \item \mla|\cite|: \cite{dijkstra1968}
  \item \mla|\citeonline|: \citeonline{dijkstra1968}
  \item \mla|\tcitep|: \tcitep{dijkstra1968}
  \item \mla|\tcitet|: \tcitet{dijkstra1968}
  \end{itemize}
\end{frame}

\begin{frame}{Pseudocódigo}
  \begin{algorithmic}
    \Procedure{Closure}{C, A}
      \State{$H \gets \emptyset$}\Comment{Direct cache}
      \For{$i \in [1, n]$}\Comment{Parallel, (dynamic,32) scheduling}
        \State{$H \gets H \cup \Call{DoImportantStuff}{i}$}
      \EndFor
    \EndProcedure
  \end{algorithmic}
  \vspace{2em}
  \begin{itemize}
  \item Pense duas vezes antes de colocar algoritmos em slides
  \end{itemize}
\end{frame}

\begin{frame}{Tabelas}
  \begin{center}
    \begin{tabular}{lccp{4cm}}
      \toprule
      Esquerda & Coluna 1    & \rotatebox{90}{90 graus}  & Parágrafo com \mla|p{4cm}|   \\
      \midrule
      $r_1$    & \cmk        &  \xmk                     & \circledi    \\
      $r_2$    &     \multicolumn{2}{c}{merged cell}     & \circledii   \\
      $r_3$    & \circlediii & \circlediv                & \circledv    \\
      $r_4$    & \circledvi  & \circledvii               & \circledviii \\
      $r_5$    & \circledix  &  x                        & y           \\
      \bottomrule 
    \end{tabular}
  \end{center}
\end{frame}

\begin{frame}{Animações com svg2pdf}
  \begin{overprint}
    \only<1>{\includegraphics{animation-1.pdf}}
    \only<2>{\includegraphics{animation-2.pdf}}
    \only<3>{\includegraphics{animation-3.pdf}}
    \only<4>{\includegraphics{animation-4.pdf}}
    \only<5>{\includegraphics{animation-5.pdf}}
  \end{overprint}
\end{frame}

\begin{frame}{Instruções para animações com svg2pdf}

  \begin{enumerate}
  \item Crie um grupo no SVG para cada frame
    \begin{itemize}
    \item Garanta que todos os frames (grupos no SVG) tenham \emph{exatamente} as mesmas dimensões
    \item Todos os frames (grupos) devem estar empilhados no eixo $y$, isto é deve existir uma reta paralela ao eixo $y$ que contenha todos os \emph{top-left} dos \emph{bounding box} de todos os grupos
    \item Não devem haver elementos no SVG fora do \emph{bounding box} definido pelo \emph{top-left} de frame com maior valor de $y$ e o \emph{bottom-right} de frame com o menor valor de $y$   
    \end{itemize}
  \item Nomeie grupos no XML do SVG
  \item No \texttt{Makefile}, adicione um \emph{target} e uma entrada na variável \texttt{frames} para cada grupo
  \item Use o ambiente \texttt{overlayarea} ou \texttt{overprint} e liste os frames com  \mla|\only<n>{\includegraphics{frame-n.pdf}}|
  \end{enumerate}
\end{frame}

\begin{frame}{pgfgantt}
  \begin{adjustbox}{max height=.9\textheight,max width=\linewidth}
    \begin{ganttchart}[vgrid,
      y unit chart=.6cm,
      title height=.6, title top shift=0,
      % title label anchor={below=-1.5},
      % bar top shift=-0.1, bar height=0.6
      ]{1}{16}
      \gantttitlelist{2018,...,2021}{4} \\
      \gantttitle{1T}{1}\gantttitle{2T}{1}\gantttitle{3T}{1}\gantttitle{4T}{1} %2018
      \gantttitle{1T}{1}\gantttitle{2T}{1}\gantttitle{3T}{1}\gantttitle{4T}{1} %2019
      \gantttitle{1T}{1}\gantttitle{2T}{1}\gantttitle{3T}{1}\gantttitle{4T}{1} %2020
      \gantttitle{1T}{1}\gantttitle{2T}{1}\gantttitle{3T}{1}\gantttitle{4T}{1} \\
      
      \ganttgroup{Courses}{1}{8} \\
      
      \ganttbar[name=mat1]{Materialization}{4}{5} 
      \ganttbar[name=mat2]{Materialization}{7}{7} \\ 
      \ganttbar[name=shacl]{SHACL}{2}{3} \\
      \ganttmilestone[name=eswc1]{ESWC}{8} \\
      \ganttbar[name=fedq]{Federated QSQ}{8}{9} \\
      \ganttmilestone[name=jrnl]{Journal}{10} \\
      \ganttbar[name=fedp]{Parallel QSQ}{10}{11} \\
      \ganttbar[name=svc]{Services}{12}{13} \\
      \ganttbar[name=qual]{Qualification}{9}{10} \\
      \ganttbar[name=thesis]{Thesis writing}{14}{16} \\
      
      % general deadlines
      % \ganttmilestone[name=iswc1]{ISWC}{10}  \ganttmilestone[name=iswc2]{}{14} \\
      % \ganttmilestone[name=eswc2]{ESWC}{12} \\
      % \ganttmilestone[name=iiwas1]{iiWAS}{7} \ganttmilestone[name=iiwas2]{}{11} \ganttmilestone[name=iiwas3]{}{15} 
      
      % links
      \ganttlink{mat2}{fedq}
      \ganttlink{fedq}{fedp}
      \ganttlink{fedq}{svc}
      \ganttlink{fedq}{jrnl}
      \ganttlink{fedq}{qual}
      \ganttlink{mat2}{eswc1}
      \ganttlink{shacl}{eswc1}
      \ganttlink{qual}{thesis}
      \ganttlink{fedp}{thesis}
      \ganttlink{svc}{thesis}
      
    \end{ganttchart}
  \end{adjustbox}
\end{frame}


%%%%%%%%%%%%%%%%%%%%%%%%%%%%
% Finalização
%%%%%%%%%%%%%%%%%%%%%%%%%%%%

\thanksframe
\referencesframe{userguide}

\begin{backup}
  \pholder{Slide de backup}
\end{backup}

\end{document}

% LocalWords:  template cls standalone GitHub Overleaf bugfixes SVGs
% LocalWords:  Re-empacotamento fontsize Makefile pdflatex imgs PDFs
% LocalWords:  shell-escape frames SVG brazil english lapesd-slides
% LocalWords:  disabletodonotes todonotes TODO's backup showbackup
% LocalWords:  hidebackup abntexcite abntex natbib nobib titleframe
% LocalWords:  frame showsections sidebar stopcountingframes default
% LocalWords:  thanksframe Thank You Questions referencesframe titulo
% LocalWords:  bibfiles pholder todonote placeholder inline addfig
% LocalWords:  opts graphicx addfiglw width Citations dijkstra Direct
% LocalWords:  Closure Parallel dynamic scheduling DoImportantStuff
% LocalWords:  lccp merged cell svg pdf
